\documentclass[10pt]{article}
\pagestyle{empty}

%%%%%%%%%%%%%%%%%%%%%%%%%%%%%%%%%%%%%%%%%%%%%%%%%%
% Do not modify the dimensions of the page
\setlength{\topmargin}{-0.5in}
\setlength{\textheight}{9.9in}
\setlength{\oddsidemargin}{0.2in}
\setlength{\textwidth}{6.0in}
% Do not modify the dimensions of the page
%%%%%%%%%%%%%%%%%%%%%%%%%%%%%%%%%%%%%%%%%%%%%%%%%%

% No paragraph indent or paragraph skip
\parindent=10pt
\parskip=0pt

\begin{document}

\centerline{\bf Advanced Database Systems (CS 764)}
\vspace{6pt}
\centerline{{\large {\bf CALF : Comparison of Attribute Layouts on Flash}}}
\vspace{6pt}
\centerline{{\bf Satish Kumar Kotha \hspace{20pt} Priyananda Shenoy}}
\vspace{6pt}
\centerline{ \{satish,shenoy\}@cs.wisc.edu }
\vspace{6pt}
\centerline{ ABSTRACT }

Most databases today use row-store column layout to physically store the data,
in spite of results showing the advantages of column-store layout. This is mostly
due to the seek cost of disk drives imposing a steep penalty on random access,
which column store relies on. With Solid State drives replacing disk drives
in database applications, there is a need to re-evaluate the question of
column layouts. Also, existing databases and evaluations tend to focus on read
cases, whereas write costs are significant for Solid State disks.

This project reexamines the question of column layout models for Flash based
databases. We propose a flexible data storage model which partitions attributes
based on a given workload, taking into account both reads and writes. Based on
the workload, we find the optimal layout of attributes into groups, each group
stored in a different page, which minimizes the total cost for that workload.
We evaluate the performance of this intelligent partitioning against n-ary and
column storage models.

\end{document}
