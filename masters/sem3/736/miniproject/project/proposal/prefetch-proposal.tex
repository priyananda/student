\documentclass[10pt]{article}
\pagestyle{empty}
\usepackage{uarial}

%%%%%%%%%%%%%%%%%%%%%%%%%%%%%%%%%%%%%%%%%%%%%%%%%%
% Do not modify the dimensions of the page
\setlength{\topmargin}{-0.5in}
\setlength{\textheight}{9.9in}
\setlength{\oddsidemargin}{0.2in}
\setlength{\textwidth}{6.0in}
% Do not modify the dimensions of the page
%%%%%%%%%%%%%%%%%%%%%%%%%%%%%%%%%%%%%%%%%%%%%%%%%%

% No paragraph indent or paragraph skip
\parindent=10pt
\parskip=0pt

\begin{document}

\centerline{\bf Advanced Operating Systems (CS 736)}
\vspace{6pt}
\centerline{{\large {\bf Enhanced File Prefetching}}}
\vspace{6pt}
\centerline{{\bf Tushar Khot \hspace{20pt} Varghese Mathew \hspace{20pt}  Priyananda Shenoy}}
\vspace{6pt}
\centerline{ \{tushar,vmathew,shenoy\}@cs.wisc.edu }

\section{Problem Summary}

Current file prefetching implementations are static, and are agnostic to file-type. We propose to
fine tune the prefetching algorithm to make informed decisions based on the type of file being
accessed. Likewise, the file-type and therefore the access pattern can be used to fine-tune the
cache eviction policy to improve cache utilization.

\section{Motivation}

We believe that the access pattern for a certain type of file generally remains the same 
across differnt files of that type. This observation can be put to use in determining ideal 
prefetch depths for different frequently encountered file-types. We expect prefetching
driven by such a file-type determined depth to outperform contemporary implementations
of prefetching which are rather static. Similarly with caching, the access pattern as suggested
by the file-type can hint at the likelihood of the block being accessed again. A cache
replacement policy driven by this can make sure that we do not cache blocks that are 
unlikely to be needed.

\section{Initial plan of approach}

The type of file determines the access pattern (like sequential, random or piecewise 
sequential) and the cache replacment policy (whether the block is likely to be accessed
again). And if the access is sequential, the rate at which the file is normally read can be used 
to compute an optimal prefetch depth for the file-type. The initial part of our project involves
gathering traces to determine file access patterns and analysing them on a per-file-type basis.
We then plan to use this information to propose a better policy for prefetching and cache replacement.

\section{Proposed Methodology and Evaluation}

\begin{itemize}
\item As a first step, we plan to instrument the linux kernel to yield a trace of file access patterns
including prefetching and cache hit/miss patterns. Through this, we desire to attain twofold objectives
	\begin{enumerate}
  \item Determine an offline, optimal prefetching and cache replacement policy, which will serve as a reference point for
     comparison in our evaluation of our own strategy.
  \item Familiarize ourselves with the prefetching code in the linux kernel.
  \end{enumerate}
\item Next we plan to gather long sample traces over real workloads. We will use these traces to 
  determine and generalize access patterns over file-types and compute the optimal prefetch 
  depth and cache eviction policy statically.
\item As the third step, we plan to incorporate the optimal depth so obtained into the linux 
  kernel and evaluate resultant performance of the system, both against the offline optimal
  strategy and the performance obtained without our modifications.
\item As a potential last step, if time permits, we will try to incorporate a feedback mechanism to 
  adjust the offline determined ideal prefetch depth.
\end{itemize}

\section{Software resources}

We propose to confine ourselves to Linux 2.6.x in this study. Additionally, we will primarily work
on the ext2 file system. However, if time permits, we may also perform the same modifications
on the IBM jfs filesystem to exclude file-system specific anomalies.

\section{Hardware resources}
We plan to run our tests on an i386 system with a SATA hard disk. We plan to use a machine in
the Crash and Burn Lab for our experiments.

\begin{thebibliography}{99}

\bibitem[1]{C1} Behdad Esfahbod.
	Preload � An Adaptive Prefetching Daemon.
	{\em Masters Thesis, Graduate Department of Computer Science. University of Toronto}, 2006.
	
\bibitem[2]{C2} Jeffrey Scott Vitter, P. Krishnan.
Optimal prefetching via data compression.
{\em Journal of the ACM, vol.43, pp. 771-793}, September 1996.

\bibitem[3]{C3} James Griffioen, Randy Appleton.
Reducing File System Latency using a
Predictive Approach
{\em USENIX 94}, 1994.

\end{thebibliography}

\end{document}